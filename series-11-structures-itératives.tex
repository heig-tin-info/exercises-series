\documentclass[10pt]{article}
\usepackage[a4paper, total={18cm, 25cm}]{geometry}
\usepackage{hexercises-common}
\usepackage{enumitem}
\usepackage{lmodern}
\usepackage{fancyhdr}
\usepackage{lastpage}
\usepackage{titlesec}

\definecolor{mygray}{rgb}{0.9,0.9,0.9}
\definecolor{points}{rgb}{0.5,0.5,0.5}

\sethexercisesinlinecolor{mygray}

\titleformat{\section}{\bfseries\sffamily\large}{Problème \thesection~-- }{0.2em}{}
\titleformat*{\subsection}{\normalsize\bfseries}
\titleformat*{\subsubsection}{\small\bfseries}
\titlespacing*{\section}{0pt}{*2}{*1}
\titlespacing*{\subsection}{0pt}{*2}{*1}

\newcommand\pts[1]{\small\color{points}\emph{(#1 pt)}}
\newcommand{\fixspacing}{\vspace{0pt plus 1filll}\mbox{}}


\begin{document}

\date{\today}
\author{INFO1-TIN-1}
\title{Série d'exercices \texttt{0x11} \\ \textbf{Structures itératives}}
\maketitle

\noindent\rule{\textwidth}{.3pt}

\begin{multicols}{2}

\section{Affichage de nombres entiers}

\begin{enumerate}[label=(\roman*)]
    \item Écrire une fonction qui affiche les nombres entiers de 1 à 100 en employant une boucle \lstinline{for}.
    \item Reécrire ce même programme avec une boucle \lstinline{while}
    \item Réécrire ce même programme avec une boucle \lstinline{do}
    \item Quelle est la structure de contrôle la plus adaptée à cette application ?
\end{enumerate}

\section{Utilisation particulière de for}

Expliquez quelle est la fonctionnalité du programme ci-dessous:

\begin{lstlisting}
#include <stdio.h>
#include <stdlib.h>

int main(void)
{
    for (size_t i = 0; i * i < 1000; i++, printf("\n"))
        printf("%d", i);
}
\end{lstlisting}

\section{Comptage}

\begin{enumerate}[label=(\roman*)]
    \item Écrire un programme qui lit une ligne de texte saisie par l'utilisateur et qui affiche le nombre de caractères saisis.

    \emph{Indication:} le dernier caractrère saisi par l'utilisateur est toujours le caractère \lstinline{LF}.

    \item Écrire une fonction qui lit une ligne de texte saisir par l'utilisateur tant qu'il n'y a que des 0 et des 1, et qui retourne l'entier correspondant au code binaire saisi.

    \item Écrire une fonction qui reçoit un nombre entier 16-bits en paramètre, et qui affiche sa représentation  en binaire à l'écran.
\end{enumerate}

\section{Testez vos connaissances}

\emph{Remarque:} ces exercices comportent des instructions \textbf{qui ne constituent pas des exemples à suivre}. Ils ont pour but de vous permettre d’analyser des formes tortueuses des instructions pour affiner votre compréhension.

Pendant les labos, veillez à écrire du code aussi limpide et bien structuré que possible.

\subsection{do...while}

Indiquer si les affirmations suivantes sont justes ou fausses. Dans les cas où elles sont fausses, expliquer ce qui serait correct. Dans une instruction \lstinline{do...while}:

\begin{enumerate}[label=(\roman*)]
    \item Les instructions de la boucle sont toujours exécutées au moins une fois.
    \item Comme un mot réservé spécifique commence et termine la boucle, on n'a pas besoin de créer un bloc lorsque l'on a plusieurs instructions.
    \item La condition se trouvant en fin de boucle, on sort de la boucle lorsque la condition est vraie.
    \item Le type de la condition peut être \lstinline{char}.
    \item Les instructions de la boucle ne peuvent pas être une autre boucle \lstinline{do...while}.
\end{enumerate}

\subsection{while}

Indiquer pour chaque groupe d'instruction ci-dessous ce qui sera affiché à l'exécution.

\begin{enumerate}[label=(\roman*)]

    \item \begin{lstlisting}
int i = 0;
while (i - 10)
{ i += 2; printf ( "%i\n", i );
}\end{lstlisting}

    \item \begin{lstlisting}
int i = 0;
while ( i - 10 )
i += 2; printf ( "%i\n", i );
\end{lstlisting}

    \item \begin{lstlisting}
int i = 0;
while ( i < 11 )
{  i += 2; printf ( "%i\n", i );
}
\end{lstlisting}

    \item \begin{lstlisting}
int i = 11;
while ( i-- )
{ printf ( "%i\n", i-- );
}
\end{lstlisting}

    \item \begin{lstlisting}
int i = 12;
while (i--)
{ printf ("%i\n", --i);
}
\end{lstlisting}

    \item \begin{lstlisting}
int i = 0;
while ( i++ < 10 )
{ printf ( "%i\n", i-- );
}
\end{lstlisting}

    \item \begin{lstlisting}
i = 1;
while ( i <= 5 )
{ printf ( "%i\n", 2 * i++ );
}
\end{lstlisting}

    \item \begin{lstlisting}
int i = 1;
while ( i != 9 )
{ printf ( "%i\n", i = i + 2 );
}
\end{lstlisting}

    \item \begin{lstlisting}
int i = 1;
while ( i < 9 )
{ printf ( "%i\n", i += 2 ); break;
}
\end{lstlisting}

    \item \begin{lstlisting}
int i = 0;
while ( i < 10 )
{ continue;
    printf ( "%i\n", i += 2 );
}
\end{lstlisting}

\end{enumerate}


\subsection{for}

Indiquer pour chaque groupe d'instruction ci-dessous ce qui sera affiché à l'exécution.

\begin{enumerate}[label=(\roman*)]

\item \begin{lstlisting}
for (int i = 'a'; i < 'd'; printf ("%i\n", ++i));\end{lstlisting}

\item \begin{lstlisting}
for (int i = 'a'; i < 'd'; printf ("%c\n", ++i));
\end{lstlisting}

\item \begin{lstlisting}
for (int i = 'a'; i++ < 'd'; printf ("%c\n", i ));
\end{lstlisting}

\item \begin{lstlisting}
for (int i = 'a'; i <= 'a' + 25; printf ("%c\n", i++ ));
\end{lstlisting}

\item \begin{lstlisting}
for (int i = 1 / 3; i ; printf("%i\n", i++ ));
\end{lstlisting}

\item \begin{lstlisting}
for (int i = 0; i != 1  ; printf("%i\n", i += 1 / 3 ));
\end{lstlisting}

\item \begin{lstlisting}
for (int i = 12, k = 1; k++ < 5 ; printf("%i\n", i-- ));
\end{lstlisting}

\item \begin{lstlisting}
for (int i = 12, k = 1; k++ < 5 ; k++,
printf("%i\n", i-- ));
\end{lstlisting}

\item \begin{lstlisting}
for (int i = 1 ; printf   ("Resultat: ") , (i+=1)<5 ;
printf("%i\n",i));
\end{lstlisting}

\end{enumerate}

\end{multicols}

\end{document}
