\documentclass[12pt,a4paper]{article}
%\usepackage[french]{babel}
\usepackage{fontspec}
%\usepackage{microtype}
\usepackage[dvipsnames]{xcolor}
\usepackage[xparse,many,skins,breakable]{tcolorbox}
\tcbuselibrary{minted}
\usepackage{minted}
\usepackage{hyperref}
\usepackage{setspace}
%\usepackage{inconsolata}
\usepackage{xparse}

% \directlua{luaotfload.add_fallback
%   ("fontfallback",
%   {
%       "NotoColorEmoji:mode=harf;",
%       "NotoNaskhArabic:mode=harf;",
%       "NotoSerifHebrew:mode=harf;",
%       "NotoSerifDevanagari:mode=harf;",
%       "NotoSerifTamil:mode=harf;",
%       "NotoSerifTibetan:mode=harf;",
%       "NotoSerifCJKkr:mode=harf;",
%       "NotoSans:mode=harf;",
%       "NotoSansSymbols2-Regular:mode=harf;",
%       "NotoSansMath-Regular:mode=harf;",
%       "NotoMusic-Regular:mode=harf;",
%     }
%   )}

% \setmainfont{Latin Modern Roman}[
%   Ligatures=TeX,
%   SmallCapsFont = Latin Modern Roman Caps,
%   RawFeature = {fallback=fontfallback},
% ]

% \setsansfont{Latin Modern Sans}[
%   RawFeature={fallback=fontfallback}
% ]

% \setmonofont{FreeMono}[
%   Scale=0.9,
%   ItalicFont={* Oblique},
%   BoldFont={* Bold},
%   BoldItalicFont={* Bold Oblique},
%   RawFeature={fallback=fontfallback}
% ]

\setstretch{1.1}

\usemintedstyle{bw}

\setminted{
    tabsize=4,
    breaklines,
    xleftmargin=0.4em,
    fontsize=\footnotesize,
    highlightcolor=gray!30,
    ignorelexererrors
}

\tcbuselibrary{minted}

\NewTCBListing{code}{ !O{text} !O{} !O{} }{%
  listing engine=minted,
  breakable, listing only, enhanced,
  colback=white, pad at break*=1mm,
  toptitle=0.15em, bottomtitle=0.1em,
  boxrule=0.15mm, arc=3pt,
  top=1mm, bottom=1mm, left=0pt, right=0pt,
  boxsep=0pt, colback=black!3, colframe=black!80,
  minted language=#1,
  minted options={#2},
  #3
}

\tcbset{
  cinline/.style={
    on line,
    tcbox raise base,
    boxsep=1pt,
    left=3pt,
    right=3pt,
    top=1pt,
    bottom=1pt,
    colback=black!3,
    colframe=black!20,
    arc=2pt,
    tcbox raise base,
  },
}

% \cinline[<lang>][<minted opts>] |code|
% \cinline[<lang>][<minted opts>]!code!   % (brace form also supported; see note)
\ExplSyntaxOn
\NewDocumentCommand{\cinline}{ o o +v }{%
  \IfNoValueTF{#1}{%
    % --- no language: literal text, no minted ---
    \tcbox[cinline]{\texttt{\detokenize{#3}}}%
  }{%
    \IfBlankTF{#1}{%
      % --- empty language []: also literal ---
      \tcbox[cinline]{\texttt{\detokenize{#3}}}%
    }{%
      % --- minted inline ---
      \IfNoValueTF{#2}{%
        \tcbox[cinline]{\mintinline{#1}{#3}}%
      }{%
        \tcbox[cinline]{\mintinline[#2]{#1}{#3}}%
      }%
    }%
  }%
}
\ExplSyntaxOff

\title{Code environment demo}
\author{HEIG-VD -- Série d'exercices}
\date{\today}

\begin{document}
\maketitle

Ce document autonome sert \'a v\'erifier le rendu des environnements \cinline!\string\code! et \cinline!\string\cinline!. Les extraits sont inspir\'es des s\'eries d'exercices en C, C++ et Python afin de couvrir plusieurs langages.

\section{Inline code}

On peut annoter une instruction \cinline!printf("Voltage: %g V\\n");! ou une condition \cinline!if valeur < 0:! directement dans le texte, ainsi qu'une commande shell comme \cinline!python3 -m pytest --maxfail=1!.

We can highlight the inline code with syntax highlighting by specifying the language: \cinline[cpp]!std::cout << "Hello, World!" << std::endl;! or \cinline[python]!def greet(name): return f"Hello, {name}"!.

\section{Code Blocks}

\subsection{No language}

\begin{code}[text]
This is a code block without a specified language.
\end{code}
\subsection{C Example}

\begin{code}[c]
#include <stdio.h>

double compute_current(double voltage, double resistance) {
  if (resistance <= 0.0) {
    fprintf(stderr, "Resistance invalide\\n");
    return -1.0;
  }
  return voltage / resistance;
}

int main(void) {
  double voltage = 5.0;
  double resistance = 220.0;
  double current = compute_current(voltage, resistance);

  if (current < 0.0) {
    return 1;
  }

  printf("Pour %.1f V et %.1f ohms, courant = %.4f A\\n",
         voltage, resistance, current);
  return 0;
}
\end{code}

\subsection{C++}
\begin{code}[cpp][linenos,highlightlines={7-10,13}]
#include <algorithm>
#include <format>
#include <iomanip>
#include <iostream>
#include <map>
#include <string>

struct Item {
  std::string name;
  int quantity;
};

int main() {
  std::map<std::string, Item> stock = {
      {"multimetre", {"multimetre", 4}},
      {"oscilloscope", {"oscilloscope", 2}},
      {"arduino", {"arduino", 18}},
  };

  auto print_item = [](const Item &item) {
    std::cout << std::left << std::setw(14) << item.name << " -> "
              << std::right << std::setw(3) << item.quantity << " pcs\\n";
  };

  std::cout << "Inventaire disponible:" << std::endl;
  for (const auto &[key, item] : stock) {
    print_item(item);
  }

  stock["arduino"].quantity -= 5;
  std::cout << std::format("Apres pret, il reste {} arduinos\\n",
                           stock["arduino"].quantity);

  return 0;
}
\end{code}

\end{document}
