\documentclass[french,a4paper,addpoints,11pt]{exam}
\usepackage{hexercises}
\usepackage{mathtools, nccmath}
\title{Programmation et algorithmique}
\seriesno{\texttt{0x08}}
\department{TIN}
\classroom{INFO1-TIN}

\setlength\answerlinelength{10 cm}
\setlength\answerskip{3ex}
\setlength\answerclearance{1.1ex}
\CorrectChoiceEmphasis{}
\renewcommand{\thepartno}{\alph{partno}}
\renewcommand{\partlabel}{\thepartno.}
\renewcommand{\arraystretch}{1.75} % expand the cells vertically

\begin{document}
\maketitle
\thispagestyle{headandfoot}

Pour vous préparer aux examens écrits, on vous propose de réaliser les exercices suivants sur papier avec un crayon et une gomme. Vous pouvez utiliser des feuilles de brouillon.

\begin{questions}

\question Écrire une fonction qui reçoit en paramètre une chaîne de caractères et qui transforme les minuscules en majuscules dans cette chaîne. Ne pas utiliser la fonction standard \CD{toupper}.

\begin{solutionordottedlines}[6cm]
\begin{lstlisting}
void to_upper(char *str) {
    int i = 0;
    while(str[i] != '\0') {
        if (str[i] >= 'a' && str[i] <= 'z')
            str[i] -= 'a' - 'A';
        i++;
    }
}
\end{lstlisting}
\end{solutionordottedlines}

\question Sans utiliser \CD{strlen}, écrire une fonction qui calcule la longueur d'une chaîne de caractère.

\begin{solutionordottedlines}[5cm]
\begin{lstlisting}
size_t strlen(char *str) {
    size_t size = 0;
    while(str[size] != '\0') {
        size++;
    }
    return size;
}
\end{lstlisting}
\end{solutionordottedlines}

\question Écrire une fonction qui retourne une valeur aléatoire entière entre a et b, vous pouvez utiliser la fonction \CD{rand()}

\begin{solutionordottedlines}[3cm]
\begin{lstlisting}
int rand_range(int a, int b) {
    return rand() % (b - a + 1) + a;
}
\end{lstlisting}
\end{solutionordottedlines}

\question Écrire une fonction qui échange deux entiers passés par référence.

\begin{solutionordottedlines}[7cm]
\begin{lstlisting}
void swap(int *a, int *b)
{
    int tmp = *a;
    *a = *b;
    *b = tmp;
}
\end{lstlisting}
\end{solutionordottedlines}

\question Écrire une fonction qui reçoit trois valeurs réelles en paramètres et qui les trie dans l'ordre croissant. Par exemple si vous avez $a=15, b=23, c=4$ après l'appel de fonction, ces valeurs vaudront : $a=4, b=15, c=23$. Vous pouvez utiliser la fonction \CD{swap} écrite plus haut pour échanger les valeurs.

\begin{solutionordottedlines}[7cm]
\begin{lstlisting}
void sort_3(double *a, double *b, double *c)
{
    if (*a > *b) swap(a, b);
    if (*a > *c) swap(a, c);
    if (*b > *c) swap(b, c);
}
\end{lstlisting}
\end{solutionordottedlines}

\question Écrire une fonction qui retourne la moyenne de trois valeurs réelles reçues en paramètres.

\begin{solutionordottedlines}[6cm]
\begin{lstlisting}
double mean_3(double a, double b, double c)
{
    return (a + b + c) / 3;
}
\end{lstlisting}
\end{solutionordottedlines}

\question Écrire une fonction qui affiche \CD{version} sur la sortie standard si un argument \CD{--version} est présent dans les arguments du programme. Cette fonction aura le prototype suivant, et la fonction main est donnée

\ifprintanswers
\begin{solution}
\begin{lstlisting}
void parse_arguments(int argc, char *argv[])
{
    for (int i = 0; i < argc; i++)
        if (strcmp(argv[i], "--version") == 0)
            printf("Version\n");
}

int main(int argc, char *argv[]) {
    parse_arguments(argc, argv);
}
\end{lstlisting}
\end{solution}
\else
\begin{lstlisting}
void parse_arguments(int argc, char *argv[]);

int main(int argc, char *argv[]) {
    parse_arguments(argc, argv);
}
\end{lstlisting}
\fillwithdottedlines{8cm}
\fi

\question Écrire une fonction qui retourne la valeur minimale d'un tableau de réels. Le prototype est le suivant :

\ifprintanswers
\begin{solution}
\begin{lstlisting}
double min(double array[], size_t size) {
    double min = array[0];
    for (int i = 1; i < size; i++)
        if (array[i] < min)
            min = array[i];
    return min;
}
\end{lstlisting}
\end{solution}
\else
\begin{lstlisting}
double min(double array[], size_t size);
\end{lstlisting}
\fillwithdottedlines{8cm}
\fi


\end{questions}

\end{document}
