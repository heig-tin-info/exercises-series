\documentclass[10pt]{article}
\usepackage[a4paper, total={18cm, 25cm}]{geometry}
\usepackage{hexercises-common}
\usepackage{enumitem}
\usepackage{lmodern}
\usepackage{fancyhdr}
\usepackage{lastpage}
\usepackage{titlesec}

\definecolor{mygray}{rgb}{0.9,0.9,0.9}
\definecolor{points}{rgb}{0.5,0.5,0.5}

\sethexercisesinlinecolor{mygray}

\titleformat{\section}{\bfseries\sffamily\large}{Problème \thesection~-- }{0.2em}{}
\titleformat*{\subsection}{\normalsize\bfseries}
\titleformat*{\subsubsection}{\small\bfseries}
\titlespacing*{\section}{0pt}{*2}{*1}
\titlespacing*{\subsection}{0pt}{*2}{*1}

\newcommand\pts[1]{\small\color{points}\emph{(#1 pt)}}
\newcommand{\fixspacing}{\vspace{0pt plus 1filll}\mbox{}}


\begin{document}

\date{\today}
\author{INFO1-TIN-1}
\title{Série d'exercices \texttt{0x11} \\ \textbf{Structures itératives}}
\maketitle

\noindent\rule{\textwidth}{.3pt}

\begin{multicols}{2}

\section{Affichage de nombres entiers}

\begin{enumerate}[label=(\roman*)]

\item \begin{lstlisting}
void print(void) {
    for (int i = 0; i < 100; i++)
        printf("%d\n", i + 1);
}
\end{lstlisting}

\item \begin{lstlisting}
void print(void) {
    int i = 0;
    while (i++ <= 100)
        printf("%d\n", i);
}
\end{lstlisting}

\item \begin{lstlisting}
void print(void) {
    int i = 0;
    do {
        printf("%d\n", i + 1);
    } while (++i < 100);
}
\end{lstlisting}

\item La boucle \lstinline{for} est la structure de contrôle la plus appropriée car le nombre d'itérations est connu à l'avance.

\end{enumerate}

\section{Utilisation particulière de for}

Le programme affiche les entiers de 1 à 1000 avec une valeur par ligne. L'utilisation de l'opérateur virgule dans la boucle for permet de ne pas imprimer le dernier retour à la ligne.

\section{Comptage}

\begin{enumerate}[label=(\roman*)]
\item \begin{lstlisting}
int main(void) {
    char buffer[1024] = {0};
    if (scanf("%1023s", buffer) != 1) return 1;
    printf("%d\n", strlen(buffer));
}
\end{lstlisting}

\item \begin{lstlisting}
int get_number(void) {
    char c;
    int value = 0;
    int k = 0;
    while (scanf("%[01]", c) == 1) {
        value |= (c == '1') << k++;
    };
    return value;
}
\end{lstlisting}

\item \begin{lstlisting}
void to_binary(uint16_t n) {
    for (int i = 16; i > 0; i--) {
        printf("%c", (n & 1 << (i - 1)) ? '1' : '0');
    }
    printf("\n");
}
\end{lstlisting}

\end{enumerate}

\section{Testez vos connaissances}

\subsection{do...while}

\begin{enumerate}[label=(\roman*)]
    \item Vrai
    \item Faux
    \item Faux; la condition est appelée condition de maintient
    \item Vrai
    \item Vrai
\end{enumerate}

\subsection{while}

Indiquer pour chaque groupe d'instruction ci-dessous ce qui sera affiché à l'exécution.

\begin{enumerate}[label=(\roman*)]

    \item \begin{lstlisting}
2
4
6
8
10\end{lstlisting}

    \item \begin{lstlisting}
10
\end{lstlisting}

    \item \begin{lstlisting}
2
4
6
8
10
12
\end{lstlisting}

    \item \begin{lstlisting}
10
8
6
4
2
0
-2
-4
...
\end{lstlisting}

    \item \begin{lstlisting}
10
8
6
4
2
0
\end{lstlisting}

    \item \begin{lstlisting}
1
1
1
1
...
\end{lstlisting}

    \item \begin{lstlisting}
2
4
6
8
10
\end{lstlisting}

    \item \begin{lstlisting}
3
5
7
9
\end{lstlisting}

    \item \begin{lstlisting}
3
\end{lstlisting}

    \item \begin{lstlisting}
Kill signal (SIGKILL)
\end{lstlisting}

\end{enumerate}


\subsection{for}

Indiquer pour chaque groupe d'instruction ci-dessous ce qui sera affiché à l'exécution.

\begin{enumerate}[label=(\roman*)]

\item \begin{lstlisting}
98
99
100\end{lstlisting}

\item \begin{lstlisting}
b
c
d
\end{lstlisting}

\item \begin{lstlisting}
b
c
d
\end{lstlisting}

\item \begin{lstlisting}
a
b
...
x
y
z
\end{lstlisting}

\item \begin{lstlisting}
(rien)
\end{lstlisting}

\item \begin{lstlisting}
0
0
0
...
\end{lstlisting}

\item \begin{lstlisting}
12
11
10
9
\end{lstlisting}

\item \begin{lstlisting}
12
11
\end{lstlisting}

\item \begin{lstlisting}
Resultat: 2
Resultat: 3
Resultat: 4
Resultat:
\end{lstlisting}

\end{enumerate}

\end{multicols}

\end{document}
