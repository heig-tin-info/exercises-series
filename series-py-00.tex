\documentclass[french,a4paper,addpoints,11pt]{exam}
\usepackage{hexercises}

\title{Introduction à Python}
\seriesno{\texttt{Py00}}
\department{TIN}
\classroom{INFO1-TIN}

\setlength\answerlinelength{10 cm}
\setlength\answerskip{3ex}
\setlength\answerclearance{1.1ex}
\CorrectChoiceEmphasis{}
\renewcommand{\thepartno}{\alph{partno}}
\renewcommand{\partlabel}{\thepartno.}
\renewcommand{\arraystretch}{1.75}

\begin{document}
\maketitle
\thispagestyle{headandfoot}

\begin{questions}

\question Quelques généralités.
\begin{parts}
    \part Qui est l'inventeur du langage Python ?
    \answerline[Guido van Rossum]

    \part Dans quelle année la première version publique du langage a-t-elle été publiée ?
    \answerline[1991]

    \part Citez au moins deux paradigmes de programmation que Python supporte.
    \answerline[Impératif, Orienté objet, Fonctionnel]

    \part Quelle implémentation de référence de Python est la plus utilisée ?
    \answerline[CPython]

    \part La traduction du code source Python en bytecode exécuté par l'interpréteur se fait lors de la \fillin[compilation] du fichier en \texttt{.pyc}.

    \part Quel est l'outil officiel pour installer des paquets Python depuis le Python Package Index ?
    \answerline[pip]

    \part Quel site communautaire dédié à la programmation propose une catégorie spécifique pour Python parmi les choix suivants ?
    \begin{checkboxes}
        \choice \url{https://www.djangoproject.com}
        \CorrectChoice \url{https://stackoverflow.com/}
        \choice \url{https://docs.python.org/}
        \choice \url{https://numpy.org}
        \choice \url{https://puzzling.stackexchange.com/}
    \end{checkboxes}
\end{parts}

\question
Associez chaque type natif Python à un exemple de valeur valide.
\begin{center}
    \begin{tabular}{p{0.35\linewidth}p{0.55\linewidth}}
        Type & Exemple \\ \hline
        \fillin[\texttt{int}] & \texttt{42} \\
        \fillin[\texttt{float}] & \texttt{3.14} \\
        \fillin[\texttt{str}] & \texttt{"Bonjour"} \\
        \fillin[\texttt{bool}] & \texttt{True} \\
        \fillin[\texttt{list}] & \texttt{[1, 2, 3]} \\
        \fillin[\texttt{tuple}] & \texttt{("x", "y")} \\
        \fillin[\texttt{dict}] & \texttt{\{"cle": 1\}} \\
    \end{tabular}
\end{center}

\question
Complétez les fragments de code suivants.
\begin{parts}
    \part Complétez la déclaration d'une fonction qui retourne la longueur d'une chaîne.
    \begin{verbatim}
    def longueur(texte):
        return \fillin[len(texte)]
    \end{verbatim}

    \part Complétez une boucle qui parcourt les éléments d'une liste.
    \begin{verbatim}
    fruits = ["pomme", "poire", "banane"]
    for \fillin[fruit] in fruits:
        print(\fillin[fruit])
    \end{verbatim}

    \part Complétez l'utilisation d'une compréhension de liste produisant les carrés de 0 à 4.
    \begin{verbatim}
    carres = [\fillin[i ** 2] for i in range(\fillin[5])]
    \end{verbatim}
\end{parts}

\question
Expliquez les termes suivants en une ou deux phrases.
\begin{parts}
    \part Interpréteur Python
    \answerline[Programme qui lit et exécute le bytecode Python en gérant la mémoire et l'environnement d'exécution.]

    \part Typage dynamique
    \answerline[Système où le type d'une variable est déterminé à l'exécution et peut varier au cours du programme.]

    \part Module
    \answerline[Fichier ou ensemble de fichiers contenant du code Python réutilisable importé via l'instruction \texttt{import}.]
\end{parts}

\end{questions}

\end{document}
