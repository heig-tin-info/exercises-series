\documentclass[10pt]{article}
\usepackage[a4paper, total={18cm, 25cm}]{geometry}
\usepackage[utf8]{inputenc}
\usepackage{fontspec}
\usepackage[french]{babel}
\usepackage[T1]{fontenc}
\usepackage{enumitem}
\usepackage{lmodern}
\usepackage{parskip}
\usepackage{xcolor}
\usepackage{listings}
\usepackage{multicol}
\usepackage{xpatch}
\usepackage{realboxes}
\usepackage{fancyhdr}
\usepackage{lastpage}
\usepackage{textcomp}
\usepackage{graphicx}
\usepackage[ttdefault=true]{AnonymousPro}
\usepackage{titlesec}

\graphicspath{{../../assets/}{../assets/}{assets/}}

\definecolor{mygray}{rgb}{0.9,0.9,0.9}
\definecolor{points}{rgb}{0.5,0.5,0.5}

\titleformat{\section}{\bfseries\sffamily\large}{Problème \thesection~-- }{0.2em}{}
\titleformat*{\subsection}{\normalsize\bfseries}
\titleformat*{\subsubsection}{\small\bfseries}
\titlespacing*{\section}{0pt}{*2}{*1}
\titlespacing*{\subsection}{0pt}{*2}{*1}

\lstset{
  language=c,
  breaklines=true,
  keywordstyle=\bfseries\color{black},
  basicstyle=\ttfamily\color{black}\fontsize{9pt}{10pt}\selectfont,
  emphstyle={\em \color{gray}},
  emph={expression, expr, type, NAME, name, expr, value, filename, label, member, type},
  keepspaces=true,
  showspaces=false,
  showtabs=true,
  tabsize=3,
  upquote=true,
  backgroundcolor=\color{mygray},
  aboveskip=2pt,
  belowskip=2pt,
  framexleftmargin=2pt,  
}

\makeatletter
\xpretocmd\lstinline{\Colorbox{mygray}\bgroup\appto\lst@DeInit{\egroup}}{}{}
\makeatother

\newcommand\pts[1]{\small\color{points}\emph{(#1 pt)}}
\newcommand{\fixspacing}{\vspace{0pt plus 1filll}\mbox{}}


\begin{document}

\date{\today}
\author{INFO1-TIN-1}
\title{Série d'exercices \texttt{0x13} \\ \textbf{Algorithmes et pointeurs}}
\maketitle

\noindent\rule{\textwidth}{.3pt}

%\begin{multicols}{2}

\section{Percolation}

Une entreprise vaudoise vous mandate pour étudier un modèle de percolation en milieu poreux. Le milieu est modélisé par une matrice aléatoire d'entiers qui détermine les sites qui peuvent être envahi par l'eau et ceux qui sont imperméables. 

Une matrice percole s'il existe un chemin d'eau allant de la ligne supérieure vers la ligne inférieure. 

La matrice suivante ne percole pas car l'eau ne pénètre pas jusqu'en bas. 

\begin{lstlisting}
1 1 2 1 2 2 1 2 2 1
2 1 1 2 2 1 1 1 2 1
2 2 2 2 2 2 1 1 2 2
2 2 1 2 2 1 1 2 2 1
1 1 0 1 2 2 2 2 2 2
0 0 1 1 1 2 2 1 1 1 
0 0 1 0 1 1 1 0 0 1 
0 1 1 1 1 1 0 1 0 0
1 0 1 1 0 1 1 0 0 0 
0 1 0 0 0 0 0 0 0 1
\end{lstlisting}

La convention est \lstinline{0} pour un vide (pore), \lstinline{1} pour un obstacle imperméable et \lstinline{2} pour l'eau.

L'eau ne peut circuler que horizontalement et verticalement.

\end{document}
