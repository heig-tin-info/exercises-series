\documentclass[10pt]{article}
\usepackage[a4paper, total={18cm, 25cm}]{geometry}
\usepackage{hexercises-common}
\usepackage{enumitem}
\usepackage{lmodern}
\usepackage{fancyhdr}
\usepackage{lastpage}
\usepackage{titlesec}

\definecolor{mygray}{rgb}{0.9,0.9,0.9}
\definecolor{points}{rgb}{0.5,0.5,0.5}

\sethexercisesinlinecolor{mygray}

\titleformat{\section}{\bfseries\sffamily\large}{Problème \thesection~-- }{0.2em}{}
\titleformat*{\subsection}{\normalsize\bfseries}
\titleformat*{\subsubsection}{\small\bfseries}
\titlespacing*{\section}{0pt}{*2}{*1}
\titlespacing*{\subsection}{0pt}{*2}{*1}

\newcommand\pts[1]{\small\color{points}\emph{(#1 pt)}}
\newcommand{\fixspacing}{\vspace{0pt plus 1filll}\mbox{}}


\begin{document}

\date{\today}
\author{INFO1-TIN-1}
\title{Série d'exercices \texttt{0x12} \\ \textbf{Algorithmes et pointeurs}}
\maketitle

\noindent\rule{\textwidth}{.3pt}

%\begin{multicols}{2}

\section{Le Crible}

Déclarez un tableau de 100 entiers que vous initialisez de 1 à 100.

Parcourez ensuite ce tableau de et remplacer tous les multiples de 2 par la valeur 0.

Faites la même chose pour les multiples de 3, 4, 5 et ainsi de suite jusqu'à 10.

Affichez à l'écran toutes les valeurs de ce tableau qui ne sont pas nulles.

\section{Passage par adresse}

Sans utiliser d'ordinateur, indiquer ce que le programme affiche à l'écran ?

\begin{lstlisting}
int foo(int* i, int *j, int *k, int p, int *q) {
    *i = *i + 2;
    j = i;
    *k += p;
    k = &p;
    *k += 4;
    q = k;
    return p;
}

int main(void) {
    int a = 11, b = 22, c =  33, d = 44, e = 55, f = d;
    e = foo(&a, &b, &c, d, &f);
    printf(
        "a. %d\n" "b. %d\n" "c. %d\n"
        "d. %d\n" "e. %d\n" "f. %d\n",
        a, b, c, d, e, f);
}
\end{lstlisting}

\section{Sphère}

Le volume d'une sphère est défini par la formule:

\[V = \frac{4}{3}\pi\cdot r^3\]

Sans utiliser d'ordinateur, écrire un programme qui invite l'utilisateur à saisir un rayon et affiche le volume de la sphère.

%\end{multicols}

\end{document}
