\documentclass[french,a4paper,addpoints,11pt]{exam}
\usepackage{hexercises}

\title{Introduction}
\seriesno{\texttt{0x00}}
\department{TIN}
\classroom{INFO1-TIN}

\setlength\answerlinelength{10 cm}
\setlength\answerskip{3ex}
\setlength\answerclearance{1.1ex}
\CorrectChoiceEmphasis{}
\renewcommand{\thepartno}{\alph{partno}}
\renewcommand{\partlabel}{\thepartno.}
\renewcommand{\arraystretch}{1.75} % expand the cells vertically

\begin{document}
\maketitle
\thispagestyle{headandfoot}

\begin{questions}

\question
Quelques généralités.
\begin{parts}
    \part Quels sont les noms des deux inventeurs du langage C ?
    \answerline[Brian Kernighan et Dennis Ritchie]

    \part Citez au moins deux paradigmes de programmation que le C supporte ?
    \answerline[Impératif, Structuré et Procédural]

    \part La \fillin[compilation] est le nom de l'étape permettant de transformer du code source en un programme exécutable ?

    \part L'unité de mesure de la quantité d'information est le \fillin[bit] qui est par définition la quantité minimale d'information transmise par un message.

    \part L'octet aujourd'hui équivalent à un \fillin[byte] offre une taille suffisante pour encoder un caractère ASCII étendu.

    \part Soit deux mains humaines dont chaque doigt peut être levé ou baissé,  quel est le nombre de configurations qui peuvent être exprimées ?
    \answerline[1024]

    \part Parmi les choix suivants, sur quel site internet peut-on poser des questions liées à la programmation en C ?
    \begin{checkboxes}
        \choice \url{http://heig-vd.ch}
        \choice \url{http://kernighan.us}
        \CorrectChoice \url{http://stackoverflow.com/}
        \choice \url{https://learnxinyminutes.com//}
        \choice \url{https://puzzling.stackexchange.com/}
    \end{checkboxes}
    \part En quel année a été inventé le langage C ?
    \begin{checkboxes}
        \choice 1941
        \choice 1969
        \CorrectChoice 1972
        \choice 1997
        \choice 2004
    \end{checkboxes}
\end{parts}

\newpage
\question Complétez la table suivante avec les valeurs qui conviennent. Utilisez la convention d'écriture C, soit le préfixe \CD{0} pour l'octal, le \CD{0b} pour le binaire et le \CD{0x} pour l'hexadécimal.

\medskip
\begin{center}
\begin{tabular}{*4{c}}
Binaire & Octal & Décimal & Hexadécimal \\ \hline
\CD{0b0000} & \fillin[000] & \fillin[0] & \fillin[0x0] \\
\CD{0b0001} & \fillin[001] & \fillin[1] & \fillin[0x1] \\
\CD{0b0010} & \fillin[002] & \fillin[2] & \fillin[0x2] \\
\fillin[0b0011] & \CD{003} & \fillin[3] & \fillin[0x3] \\
\CD{0b0100} & \fillin[004] & \fillin[4] & \fillin[0x4] \\
\CD{0b0101} & \fillin[005] & \fillin[5] & \fillin[0x5] \\
\CD{0b0110} & \fillin[006] & \fillin[6] & \fillin[0x6] \\
\CD{0b0111} & \fillin[007] & \fillin[7] & \fillin[0x7] \\
\CD{0b1000} & \fillin[010] & \fillin[8] & \fillin[0x8] \\
\CD{0b1001} & \fillin[011] & \fillin[9] & \fillin[0x9] \\
\fillin[0b1010] & \fillin[012] & \CD{10} & \fillin[0xA] \\
\CD{0b1011} & \fillin[013] & \fillin[11] & \fillin[0xB] \\
\fillin[0b1100] & \fillin[014] & \fillin[12] & \CD{0xC} \\
\CD{0b1101} & \fillin[015] & \fillin[13] & \fillin[0xD] \\
\CD{0b1110} & \fillin[016] & \fillin[14] & \fillin[0xE] \\
\CD{0b1111} & \fillin[017] & \fillin[15] & \fillin[0xF] \\

\end{tabular}
\end{center}

\question
Afin de préserver la méthode d'addition et de soustraction standard avec retenue, la technique du complément à $(b - 1) + 1$ est utilisée. Ainsi en binaire la base étant $2$, on nomme la technique le \guillemotleft Complément à 2\guillemotright~qu'il faut lire \guillemotleft Complément à (2 - 1), plus 1\guillemotright.
Dans les valeurs ci-dessous la base est exprimée en indice p. ex. $253_{64}$ exprimé en base 64.

\begin{parts}
    \part Quel est le complément à 1 du nombre 8-bit $10011011_2$ ?
    \answerline[$01100100_2$]

    \part Quel est le complément à 5 du nombre 8-bit $124530_6$ ?
    \answerline[$431025_6$]

    \part Quelle est la représentation binaire signée 8-bit du nombre  $13_{10}$ ?
    \answerline[$00001101_2$]

\end{parts}

\newpage
\question
Complétez le tableau ci-dessous qui comporte dans chaque ligne une valeur 8-bits à exprimer en utilisant les autres systèmes de numération vus en cours. \textbf{Note : ne pas utiliser la calculatrice}.
\medskip
\begin{center}
    \begin{tabular}{*5{c}}
    Binaire & Octal & Décimal non signé & Décimal signé & Hexadécimal \\ \hline
    \CD{0b00000001} & \fillin[0001] & \fillin[1] & \fillin[1] & \fillin[0x1] \\
    \fillin[\CD{0b11111111}] & \fillin[0377] & \fillin[255] & $-1$ & \fillin[0xFF] \\
    \fillin[\CD{0b11111001}] & \fillin[0371] & \fillin[249] & \fillin[$-7$] & \CD{0xF9} \\
    \CD{0b10101011} & \fillin[0253] & \fillin[171] & \fillin[-85] & \fillin[0xAB] \\
    \fillin[\CD{0b00001100}] & \CD{014} & \fillin[12] & \fillin[12] & \fillin[0xC] \\
    \end{tabular}
\end{center}

\end{questions}

\end{document}
