\documentclass[french,a4paper,addpoints,11pt]{exam}
\usepackage{hexercises}
\usepackage{mathtools, nccmath}
\title{Numération et Arguments}
\seriesno{\texttt{0x01}}
\department{TIN}
\classroom{INFO1-TIN}

\setlength\answerlinelength{10 cm}
\setlength\answerskip{3ex}
\setlength\answerclearance{1.1ex}
\CorrectChoiceEmphasis{}
\renewcommand{\thepartno}{\alph{partno}}
\renewcommand{\partlabel}{\thepartno.}
\renewcommand{\arraystretch}{1.75} % expand the cells vertically

\begin{document}
\maketitle
\thispagestyle{headandfoot}

\begin{questions}

\question
Quelques généralités.
\begin{parts}
    \part Comment écrire en C une constante littérale hexadécimale correspondant à la valeur $1ab4_{16}$ ?
    \answerline[0x1ab4]

    \part Comment écrire en C une constante littérale octale représentant la valeur $2642_{8}$ ?
    \answerline[02642]

    \part Bien qu'il ne soit pas normalisé dans C11, le préfixe \fillin[0b] est utilisé pour représenter une constante littérale binaire ?

    \part Le signe d'un nombre en complément à deux peut être connu en observant son bit de poids \fillin[fort] tandis que le bit de poids \fillin[faible] renseigne sur sa parité (pair ou impair).

    \part Le programme suivant est appelé. Combien d'arguments le programme reçoit-il ?

\begin{lstlisting}
    ./a.out 1 2 3
\end{lstlisting}
\answerline[4 arguments]


    \part Comment s'appelle le flux qui permet à un programme de recevoir des informations par exemple saisies au clavier ?
    \answerline[L'entrée standard]

    \part Combien de bytes y a-t-il dans un Mebibyte (MiB) ?
    \answerline[$1024\cdot 1024 = 1'048'576~\text{Bytes}$]

    \part Combien de bits faut-il pour représenter un Gigabyte (GB) de données ?
    \answerline[$\lceil \log_2\left(1 \cdot 10^9 \right)\rceil \approx \lceil 29.89 \rceil = 30~\text{bits}$]
\end{parts}

\question Effectuer les additions suivantes à la main en binaire sur 8 bits
\begin{parts}
\part $1 + 51$
\ifprintanswers
\begin{solution}
\begin{lstlisting}
        11
  00000001 = 0x01 = 1
+ 00110011 = 0x33 = 51
----------
  00110100 = 0x34 = 52
\end{lstlisting}
\end{solution}
\else
\makeemptybox{3cm}
\fi

\part $51 - 7$
\ifprintanswers
\begin{solution}
\begin{lstlisting}
 1111  11
  00110011 = 0x33 = 51
+ 11111001 = (~00000111 + 1) = 0xF9 = -7
----------
  00101100 = 0x2C = 44
\end{lstlisting}
\end{solution}
\else
\makeemptybox{3cm}
\fi
\part $204 + 51$
\ifprintanswers
\begin{solution}
\begin{lstlisting}

  11001100 = 0xCC = 204
+ 00110011 = 0x33 = 51
----------
  11111111 = 0xFF = 255
\end{lstlisting}
\end{solution}
\else
\makeemptybox{3cm}
\fi
\part $204 + 204$
\ifprintanswers
\begin{solution}
\begin{lstlisting}
     11
  11001100 = 0xCC = 204
+ 11001100 = 0xCC = 204
----------
  10011000 = 0x98 = (408-256) = 152
\end{lstlisting}
\end{solution}
\else
\makeemptybox{3cm}
\fi
\end{parts}

\question Écrire un programme qui lit le premier nombre passé en argument depuis la ligne de commande et l'affiche à l'écran.

\ifprintanswers
\begin{solution}
    \begin{lstlisting}
    int main(int argc, char *argv[]) {
        printf("Votre nombre est : %d\n", atoi(argv[1]));
    }
    \end{lstlisting}
\end{solution}
\else
\fillwithdottedlines{4cm}
\fi

\question Écrire un programme qui affiche explicitement la somme de deux nombres.
Ces deux nombres sont déclarés dans des variables entières nommées \CD{a} et \CD{b}.

Voici ce que produit le programme :

\begin{lstlisting}
./a.out
42 + 23 = 65
\end{lstlisting}

\ifprintanswers
\begin{solution}
\begin{lstlisting}
int main() {
    int a = 42;
    int b = 23;
    printf("%d + %d = %d", a, b, a + b);
}
\end{lstlisting}
\end{solution}
\else
\fillwithdottedlines{4cm}
\fi

\end{questions}

\end{document}
