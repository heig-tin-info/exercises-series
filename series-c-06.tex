\documentclass[french,a4paper,addpoints,11pt]{exam}
\usepackage{hexercises}
\usepackage{mathtools, nccmath}
\title{Opérateurs et embranchements}
\seriesno{\texttt{0x06}}
\department{TIN}
\classroom{INFO1-TIN}

\setlength\answerlinelength{10 cm}
\setlength\answerskip{3ex}
\setlength\answerclearance{1.1ex}
\CorrectChoiceEmphasis{}
\renewcommand{\thepartno}{\alph{partno}}
\renewcommand{\partlabel}{\thepartno.}
\renewcommand{\arraystretch}{1.75} % expand the cells vertically

\begin{document}
\maketitle
\thispagestyle{headandfoot}

\begin{questions}

\question Opérateurs Combinés : en reprenant à chaque fois les valeurs suivntes, calculer les valeurs de \CD{i}, \CD{j} et lorsque cela s'applique \CD{z} après l'exécution des instructions suivantes :

\begin{lstlisting}
int i = 1; j = 3;
int z;
\end{lstlisting}

\begin{center}
\begin{tabular}{l|*3{c}}
  Expression & \CD{i} & \CD{j} & \CD{z} \\ \hline
  \CD{i += j} & \fillin[4] & \fillin[3] & \fillin[?] \\
  \CD{i += -j} & \fillin[-2] & \fillin[3] & \fillin[?] \\
  \CD{i -= j} & \fillin[-2] & \fillin[3] & \fillin[?] \\
  \CD{i -= -j} & \fillin[4] & \fillin[3] & \fillin[?] \\
  \CD{i *= j} & \fillin[3] & \fillin[3] & \fillin[?] \\
  \CD{i *= -j} & \fillin[-3] & \fillin[3] & \fillin[?] \\
  \CD{i /= j} & \fillin[0] & \fillin[3] & \fillin[?] \\
  \CD{z = i * j == 6} & \fillin[1] & \fillin[3] & \fillin[0] \\
  \CD{z = i++ * j == 6} & \fillin[2] & \fillin[3] & \fillin[0] \\
  \CD{z = ++i * j == 6} & \fillin[2] & \fillin[3] & \fillin[1] \\
\end{tabular}
\end{center}

\question Opérateur ternaire
\begin{parts}

\part Simplifiez l'expression suivante
\begin{lstlisting}
z = (a > b ? a : b) + (a <= b ? a : b) ;
\end{lstlisting}

\ifprintanswers
\begin{solution}
\begin{lstlisting}
z = a + b;
\end{lstlisting}
\end{solution}
\else
\fillwithdottedlines{1cm}
\fi

\part Soit variable \CD{n} est de type \CD{int}. Écrire une expression unique qui prend la valeur:

\begin{enumerate}
  \item \CD{-1} si \CD{n} est négatif
  \item \CD{0} si \CD{n} est nul
  \item \CD{1} si \CD{n} est positif
\end{enumerate}

\ifprintanswers
\begin{solution}
\begin{lstlisting}
n < 0 ? -1 : (n == 0) ? 0 : 1
\end{lstlisting}
\end{solution}
\else
\fillwithdottedlines{1cm}
\fi

\end{parts}

\newpage
\question Opérateurs incorrects

Soit les déclarations suivantes, indiquez pourquoi les propositions suivantes sont incorrectes :
\begin{lstlisting}
double f, g = 7;
int i, j = j;
\end{lstlisting}

\begin{parts}
\part
\begin{lstlisting}
int(f) + 1.9
\end{lstlisting}

\ifprintanswers
\begin{solution}
La coercition de type s'écrit \CD{(int)f} et non \CD{int(f)}.
\end{solution}
\else
\fillwithdottedlines{1.5cm}
\fi

\part
\begin{lstlisting}
i = 1 + j = j / 2
\end{lstlisting}

\ifprintanswers
\begin{solution}
L'opérateur \CD{=} est moins prioritaire que \CD{+}. En conséquence, cette expression tente d'affecter \CD{j / 2} à \CD{1 + j}. Or, une affectation n'est possible que si l'expression à gauche est une variable, ce qui n'est pas le cas ici.
\end{solution}
\else
\fillwithdottedlines{1.5cm}
\fi

\part
\begin{lstlisting}
f = g << 2
\end{lstlisting}

\ifprintanswers
\begin{solution}
Le décalage de bits est un opérateur de type \CD{int}, et non un opérateur de type \CD{double}.
\end{solution}
\else
\fillwithdottedlines{1.5cm}
\fi

\part
\begin{lstlisting}
i = ++j++
\end{lstlisting}

\ifprintanswers
\begin{solution}
Cette expression essaie d'exécuter l'opérateur de post-incrémentation à \CD{(++j)} or cette expression est évalué comme une valeur, et non une variable.
\end{solution}
\else
\fillwithdottedlines{1.5cm}
\fi

\part
\begin{lstlisting}
i++ = ++j
\end{lstlisting}

\ifprintanswers
\begin{solution}
La partie gauche de l'opérateur \CD{i++} est évaluée comme une valeur, et n'est donc pas assignable. On dit que le membre de gauche n'est pas une \emph{lvalue}.
\end{solution}
\else
\fillwithdottedlines{1.5cm}
\fi

\end{parts}

\question Indiquez pour chaque groupe d'instruction ci-dessous si l'expression est correcte ou non. Sinon, expliquer pourquoi.

\begin{lstlisting}
int i;
assert(scanf("%d", &i) == 1);
\end{lstlisting}

\begin{parts}

\part
\begin{lstlisting}
if (!(i < 8) && !(i > 8)) then printf("i vaut 8\n");
\end{lstlisting}

\ifprintanswers
\begin{solution}
Incorrect : une erreur apparaît à la compilation, le mot \emph{then} n'est pas valide en C.
\end{solution}
\else
\fillwithdottedlines{1.5cm}
\fi

\part
\begin{lstlisting}
if (!(i < 8) && !(i > 8)) printf("i "); printf("vaut 8\n");
\end{lstlisting}

\ifprintanswers
\begin{solution}
Incorrect : la première instruction est correctement exécutée mais la seconde s'exécute inconditionnellement.
\end{solution}
\else
\fillwithdottedlines{1.5cm}
\fi

\part
\begin{lstlisting}
if !(i < 8) && !(i > 8) printf("i vaut 8\n");
\end{lstlisting}

\ifprintanswers
\begin{solution}
Incorrect : l'expression est mal formée, la condition d'embranchement après le \emph{if} doit être entre parenthèses.
\end{solution}
\else
\fillwithdottedlines{1.5cm}
\fi

\part
\begin{lstlisting}
if (!(i < 8) && !(i > 8)) printf("i vaut 8\n");
\end{lstlisting}

\ifprintanswers
\begin{solution}
Correct !
\end{solution}
\else
\fillwithdottedlines{1.5cm}
\fi

\part
\begin{lstlisting}
if (i = 8) printf("i vaut 8\n");
\end{lstlisting}

\ifprintanswers
\begin{solution}
Incorrect : affiche que \CD{i} vaut 8 dans tous les cas.
\end{solution}
\else
\fillwithdottedlines{1.5cm}
\fi

\part
\begin{lstlisting}
if (i & (1 << 3)) printf("i vaut 8\n");
\end{lstlisting}

\ifprintanswers
\begin{solution}
Correct !
\end{solution}
\else
\fillwithdottedlines{1.5cm}
\fi

\part
\begin{lstlisting}
if (i ^ 8) printf("i vaut 8\n");
\end{lstlisting}

\ifprintanswers
\begin{solution}
Incorrect : affiche \emph{i vaut 8} pour tous les cas sauf lorsque \CD{i} vaut 8 !
\end{solution}
\else
\fillwithdottedlines{1.5cm}
\fi

\part
\begin{lstlisting}
if (i - 8) printf("i vaut 8\n");
\end{lstlisting}

\ifprintanswers
\begin{solution}
Incorrect : affiche \emph{i vaut 8} pour tous les cas sauf lorsque \CD{i} vaut 8 !
\end{solution}
\else
\fillwithdottedlines{1.5cm}
\fi

\part
\begin{lstlisting}
if (i == 1 << 3) printf ("i vaut 8\n");
\end{lstlisting}

\ifprintanswers
\begin{solution}
Correct !
\end{solution}
\else
\fillwithdottedlines{1.5cm}
\fi

\part
\begin{lstlisting}
if (!((i < 8) || (i > 8))) printf("i vaut 8\n");
\end{lstlisting}

\ifprintanswers
\begin{solution}
Correct !
\end{solution}
\else
\fillwithdottedlines{1.5cm}
\fi
\end{parts}

\question Que voyez-vous sur la sortie standard ?
\begin{lstlisting}
#include <stdio.h>
int main() {
    int x = 2;
    int y = ++x * ++x;
    printf("%d%d", x, y);
    x = 2;
    y = x++ * ++x;
    printf("%d%d", x, y);
}
\end{lstlisting}

\ifprintanswers
\begin{solution}
41648
\end{solution}
\else
\fillwithdottedlines{1cm}
\fi

\end{questions}

\end{document}
