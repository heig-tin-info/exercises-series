\documentclass[10pt]{article}
\usepackage[a4paper, total={18cm, 25cm}]{geometry}
\usepackage{hexercises-common}
\usepackage{enumitem}
\usepackage{lmodern}
\usepackage{fancyhdr}
\usepackage{lastpage}
\usepackage{titlesec}

\definecolor{mygray}{rgb}{0.9,0.9,0.9}
\definecolor{points}{rgb}{0.5,0.5,0.5}

\sethexercisesinlinecolor{mygray}

\titleformat{\section}{\bfseries\sffamily\large}{Problème \thesection~-- }{0.2em}{}
\titleformat*{\subsection}{\normalsize\bfseries}
\titleformat*{\subsubsection}{\small\bfseries}
\titlespacing*{\section}{0pt}{*2}{*1}
\titlespacing*{\subsection}{0pt}{*2}{*1}

\newcommand\pts[1]{\small\color{points}\emph{(#1 pt)}}
\newcommand{\fixspacing}{\vspace{0pt plus 1filll}\mbox{}}


\begin{document}

\date{\today}
\author{HEIG-TIN-INFO2}
\title{Série d'exercices \texttt{0x23} \\ \textbf{Allocation Dynamique}}
\maketitle

\noindent\rule{\textwidth}{.3pt}

\begin{multicols}{2}

\section{Tableau multidimensionnel}

On souhaite charger en mémoire une image en niveau de gris de 1024x1024 pixels. La profondeur est de 8-bits, c'est à dire que le niveau de gris est stocké sur 8-bits avec la valeur \lstinline{0x00} pour le noir et \lstinline{0xff} pour le blanc. 

Qu'écrire pour allouer dynamiquement ce tableau ?

\section{Image BMP}

On souhaite charger en mémoire une image en niveau de gris de NxM pixels. La profondeur est de 24-bits, c'est à dire que chaque couleur est stockée sur 8-bits dans l'ordre rouge, vert, bleu. 

On demande d'écrire le contenu de la fonction dont le prototype est le suivant :  

\begin{lstlisting}
char *image_allocate(size_t width, size_t height);
\end{lstlisting}

Comment écrire la fonction \lstinline{image_allocate} ?

\section{Tableau de structures}

Considérons la déclaration suivante:

\begin{lstlisting}
#define MAX_LENGTH 64

typedef struct Article {
    char name[MAX_LENGTH]
    int price; // Stored in cents
    int quantity;    
} Article;
\end{lstlisting}
    
\begin{enumerate}[label=(\roman*)]
    \item Écrire la déclaration permettant d'allouer un tableau de 100 \lstinline{Article}.
    \item Doubler dynamiquement la taille de ce tableau.
    \item Libérer l'espace méméoire utilisé.
\end{enumerate}

\end{multicols}

\end{document}
