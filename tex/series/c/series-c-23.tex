\documentclass{worksheet}
\usepackage[french]{babel}
\graphicspath{{../../assets/}{../assets/}{assets/}}
\usepackage{mathtools, nccmath}
\usepackage{tabularx}
\title{Les pointeurs}
\course{INFO2-TIN}
\seriesnumber{35}

\begin{document}
\maketitle

\begin{questions}

\question Complétez le tableau suivant en indiquant la valeur des variables après chaque instruction.

Pour indiquer que la variable \cinline!p! contient l'adresse \cinline!k!, on notera \cinline!&a! dans la colonne \cinline!p!.

Pour indiquer qu'une variable est non initialisée ou que son contenu n'est pas connu, la case sera laissée vide.

\setlength\fillinlinelength{1.5cm}

\medskip
\begin{center}
\begin{tabularx}{\textwidth}{Xccccc}
Instruction & \cinline!a! & \cinline!b! & \cinline!c! & \cinline!p! & \cinline!q! \\ \hline
\canswer{int a = 1;}  & \fillin[1] & \fillin[ ] & \fillin[ ] & \fillin[ ] & \fillin[ ] \\
\canswer{int b = 2;}  & \fillin[1] & \fillin[2] & \fillin[ ] & \fillin[ ] & \fillin[ ] \\
\canswer{int c = 3;}  & \fillin[1] & \fillin[2] & \fillin[3] & \fillin[ ] & \fillin[ ] \\
\canswer{int *p = &a;} & \fillin[1] & \fillin[2] & \fillin[3] & \fillin[\&a] & \fillin[ ] \\
\canswer{int *q = &c;} & \fillin[1] & \fillin[2] & \fillin[3] & \fillin[\&a] & \fillin[\&c] \\
\canswer{*p=(*q)++;}  & \fillin[3] & \fillin[2] & \fillin[4] & \fillin[\&a] & \fillin[\&c] \\
\canswer{p = q;}  & \fillin[3] & \fillin[2] & \fillin[4] & \fillin[\&c] & \fillin[\&c] \\
\canswer{q=&b;}  & \fillin[3] & \fillin[2] & \fillin[4] & \fillin[\&c] & \fillin[\&b] \\
\canswer{*p-=*q;}  & \fillin[3] & \fillin[2] & \fillin[2] & \fillin[\&c] & \fillin[\&b] \\
\canswer{++*q;}  & \fillin[3] & \fillin[3] & \fillin[2] & \fillin[\&c] & \fillin[\&b] \\
\canswer{*p*=*q;}  & \fillin[3] & \fillin[3] & \fillin[6] & \fillin[\&c] & \fillin[\&b] \\
\canswer{a=++*q**p;}  & \fillin[24] & \fillin[4] & \fillin[6] & \fillin[\&c] & \fillin[\&b] \\
\canswer{p=&a;}  & \fillin[24] & \fillin[4] & \fillin[6] & \fillin[\&a] & \fillin[\&b] \\
\canswer{*q=*p/=*q;}  & \fillin[6] & \fillin[6] & \fillin[6] & \fillin[\&a] & \fillin[\&b] \\
\canswer{int *r[3] = {&a, &b, &c};}  & \fillin[6] & \fillin[6] & \fillin[6] & \fillin[\&a] & \fillin[\&b] \\
\end{tabularx}
\end{center}

\clearpage
\question Intéressons-nous à l'arithmétique de pointeurs. On considère \cinline!p! un pointeur qui \emph{pointe} sur un tableau \cinline!a!:

\begin{code}[c]
int a[] = {4, 8, 15, 16, 23, 42, 66, 104, 162};
int *p = a;
\end{code}

Quelles sont les valeurs ou adresses que fournissent ces expressions ?

\begin{parts}

\setlength\fillinlinelength{8cm}
\part \cinline!*p+2!\hfill\fillin[6]
\part \cinline!*(p+2)!\hfill\fillin[15]
\part \cinline!&a[4]-3!\hfill\fillin[L'adresse de l'élément \canswer{a[1]}]
\part \cinline!a + 3!\hfill\fillin[L'adresse de l'élément \canswer{a[3]}]
\part \cinline!&a[7]-p!\hfill\fillin[7]
\part \cinline!p+(*p-2)!\hfill\fillin[L'adresse de l'élément \canswer{a[2]}]
\part \cinline!*(p+*(p+4)-a[3])!\hfill\fillin[104]
\part \cinline!(p+1)[2]!\hfill\fillin[16]
\part \cinline!5[p] // wtf!\hfill\fillin[42]
\part \cinline!(uintptr_t)(p + 3) - (uintptr_t)a!\hfill\fillin[12 (bytes)]
\part \cinline!(&a)[1][-1]!\hfill\fillin[162]
\end{parts}

\question Écrire une fonction qui respecte le prototype ci-dessous. Cette fonction copie une chaîne de caractère de la source vers la destination.

\begin{code}[c]
void strcpy(char *dest, const char *src);
\end{code}

Ne pas utiliser de boucle for, ni d'accès tableaux (\cinline!a[b]!). Utilisez une boucle \cinline!while! et l'arithmétique de pointeurs.

\begin{solutionordottedlines}[8cm]
\begin{code}[c]
void strcpy(char *dest, const char *src)
{
    char *p = dest;
    while (*src)
        *p++ = *src++;
    *p = '\0';
}
\end{code}
\end{solutionordottedlines}

\end{questions}

\end{document}
