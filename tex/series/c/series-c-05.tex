\documentclass[french,a4paper,addpoints,11pt]{exam}
\usepackage{hexercises}
\usepackage{mathtools, nccmath}
\title{Opérateurs logiques}
\seriesno{\texttt{0x05}}
\department{TIN}
\classroom{INFO1-TIN}

\setlength\answerlinelength{10 cm}
\setlength\answerskip{3ex}
\setlength\answerclearance{1.1ex}
\CorrectChoiceEmphasis{}
\renewcommand{\thepartno}{\alph{partno}}
\renewcommand{\partlabel}{\thepartno.}
\renewcommand{\arraystretch}{1.75} % expand the cells vertically

\begin{document}
\maketitle
\thispagestyle{headandfoot}

\begin{questions}

\question Opérateurs de relation et opérateurs logiques

\begin{parts}
\part Ajouter à l'expression suivante toutes les parenthèses montrant l'ordre d'exécution des opérations.

\ifprintanswers
\begin{solution}
\begin{lstlisting}
double x, y;
int z = ((((x >= 0) && (x <= 20)) && (y > x)) || ((y == 50) && (x == (1 * 2))) || (y == (30 + 30)));
\end{lstlisting}
\end{solution}
\else
\begin{lstlisting}
double x, y;
int z = x >= 0 && x <= 20 && y > x || y == 50 && x == 1 * 2 || y == 30 + 30;
\end{lstlisting}
\fi

\part Quelle est la valeur de \CD{z} évaluées avec les valeurs suivantes ?

\begin{parts}
\part \CD{x = -1.0; y = 60.0;} \hfill \fillin[1][5cm]
\part \CD{x = 0.0 ; y = 1.0;} \hfill \fillin[1][5cm]
\part \CD{x = 19.0 ; y = 1.0;} \hfill \fillin[0][5cm]
\part \CD{x = 0.0 ; y = 50.0;} \hfill \fillin[1][5cm]
\part \CD{x = 2.0 ; y = 50.0;} \hfill \fillin[1][5cm]
\part \CD{x = -10.0 ; y = 60.0;} \hfill \fillin[1][5cm]
\end{parts}

\end{parts}

\question Cas particuliers

\begin{parts}
\part Que vaut \CD{i} ? \hfill \fillin[\CD{i} vaut 32768][5cm]
\begin{lstlisting}
uint16_t i = 32767;
i++;
\end{lstlisting}

\part Que vaut \CD{i} ? \hfill \fillin[\CD{i} vaut 0.][5cm]
\begin{lstlisting}
int16_t i = 0;
--i; i--; i++; ++i;
\end{lstlisting}

\part Que vaut \CD{i} ? \hfill \fillin[\CD{i} vaut \CD{'D'}, soit 68.][5cm]
\begin{lstlisting}
short i = 'A' > 'B' ? 'C' : 'D';
\end{lstlisting}

\part Que valent \CD{i}, \CD{j} et \CD{k} ? \hfill \fillin[i: 1, j: 0, k: 5][5cm]
\begin{lstlisting}
short i = 0, j = 1, k;
k = (k = 5, i++) >= j ? i++ : --j;
\end{lstlisting}

\part Que valent \CD{i}, \CD{j} et \CD{k} ? \hfill \fillin[i: 3, j: 1, k: 8][5cm]
\begin{lstlisting}
short i = 2, j = 1, k;
k = i >= j << 1 ? i++ << 2 : --j << 3;
\end{lstlisting}

\end{parts}

\question Calcul de masques : que vaut \CD{m} en binaire ?

\begin{lstlisting}
char m, n = 2;
char d = 0x55, e = 0xaa;
\end{lstlisting}

\begin{parts}
  \part \CD{m = 1 << n;} \hfill \fillin[0b00000100][5cm]
  \part \CD{m = ~(1 << n);} \hfill \fillin[0b11111011][5cm]
  \part \CD{m = d | (1 << n);} \hfill \fillin[0b01010101][5cm]
  \part \CD{m = e | (1 << n);} \hfill \fillin[0b10101110][5cm]
  \part \CD{m = d ^ (1 << n);} \hfill \fillin[0b01010001][5cm]
  \part \CD{m = e ^ (1 << n);} \hfill \fillin[0b10101110][5cm]
  \part \CD{m = d & ~(1 << n);} \hfill \fillin[0b01010001][5cm]
  \part \CD{m = e & ~(1 << n);} \hfill \fillin[0b10101010][5cm]
\end{parts}

\question Résoudre le problème avec une ligne de code.

\begin{parts}
\part Mettre à 1 (\emph{set}) le n-ième bit de la variable \CD{x}.

\ifprintanswers
\begin{solution}
x |= 1 << n;
\end{solution}
\else
\fillwithdottedlines{1cm}
\fi

\part Mettre à 0 (\emph{clear}) le n-ième bit de la variable \CD{x}.

\ifprintanswers
\begin{solution}
\begin{lstlisting}
x &= ~(1 << n);
\end{lstlisting}
\end{solution}
\else
\fillwithdottedlines{1cm}
\fi

\part Inverser (\emph{toggle}) le n-ième bit de la variable \CD{x}.

\ifprintanswers
\begin{solution}
\begin{lstlisting}
x ^= 1 << n;
\end{lstlisting}
\end{solution}
\else
\fillwithdottedlines{1cm}
\fi
\end{parts}

\question Écrire une fonction en C qui inverse les chiffres d'un entier. Utilisez le prototype suivant :

\begin{lstlisting}
int32_t reverse(int32_t num);
\end{lstlisting}

Exemple : 123 donne 321 et 208478933 donne 339874802.

\ifprintanswers
\begin{solution}
\begin{lstlisting}
int32_t reverse(int32_t num) {
  int32_t reversed = 0;

  while (num != 0) {
    // Extraire le dernier chiffre
    int32_t digit = num % 10;

    // Ajouter ce chiffre à la valeur retournée
    reversed = reversed * 10 + digit;

    // Retirer le dernier chiffre
    num /= 10;
  }

  return reversed;
}
\end{lstlisting}
\end{solution}
\else
\fillwithdottedlines{10cm}
\fi


\end{questions}

\end{document}
