\documentclass{worksheet}
\usepackage[french]{babel}
\graphicspath{{../../assets/}{../assets/}{assets/}}
\usepackage{mathtools, nccmath}
\title{Programmation et algorithmique}
\course{INFO1-TIN}
\seriesnumber{8}

\begin{document}
\maketitle

Pour vous préparer aux examens écrits, on vous propose de réaliser les exercices suivants sur papier avec un crayon et une gomme. Vous pouvez utiliser des feuilles de brouillon.

\begin{questions}

\question Écrire une fonction qui reçoit en paramètre une chaîne de caractères et qui transforme les minuscules en majuscules dans cette chaîne. Ne pas utiliser la fonction standard \cinline!toupper!.

\begin{solutionordottedlines}[6cm]
\begin{code}[c]
void to_upper(char *str) {
    int i = 0;
    while(str[i] != '\0') {
        if (str[i] >= 'a' && str[i] <= 'z')
            str[i] -= 'a' - 'A';
        i++;
    }
}
\end{code}
\end{solutionordottedlines}

\question Sans utiliser \cinline!strlen!, écrire une fonction qui calcule la longueur d'une chaîne de caractère.

\begin{solutionordottedlines}[5cm]
\begin{code}[c]
size_t strlen(char *str) {
    size_t size = 0;
    while(str[size] != '\0') {
        size++;
    }
    return size;
}
\end{code}
\end{solutionordottedlines}

\question Écrire une fonction qui retourne une valeur aléatoire entière entre a et b, vous pouvez utiliser la fonction \cinline!rand()!

\begin{solutionordottedlines}[3cm]
\begin{code}[c]
int rand_range(int a, int b) {
    return rand() % (b - a + 1) + a;
}
\end{code}
\end{solutionordottedlines}

\question Écrire une fonction qui échange deux entiers passés par référence.

\begin{solutionordottedlines}[7cm]
\begin{code}[c]
void swap(int *a, int *b)
{
    int tmp = *a;
    *a = *b;
    *b = tmp;
}
\end{code}
\end{solutionordottedlines}

\question Écrire une fonction qui reçoit trois valeurs réelles en paramètres et qui les trie dans l'ordre croissant. Par exemple si vous avez $a=15, b=23, c=4$ après l'appel de fonction, ces valeurs vaudront : $a=4, b=15, c=23$. Vous pouvez utiliser la fonction \cinline!swap! écrite plus haut pour échanger les valeurs.

\begin{solutionordottedlines}[7cm]
\begin{code}[c]
void sort_3(double *a, double *b, double *c)
{
    if (*a > *b) swap(a, b);
    if (*a > *c) swap(a, c);
    if (*b > *c) swap(b, c);
}
\end{code}
\end{solutionordottedlines}

\question Écrire une fonction qui retourne la moyenne de trois valeurs réelles reçues en paramètres.

\begin{solutionordottedlines}[6cm]
\begin{code}[c]
double mean_3(double a, double b, double c)
{
    return (a + b + c) / 3;
}
\end{code}
\end{solutionordottedlines}

\question Écrire une fonction qui affiche \cinline!version! sur la sortie standard si un argument \cinline!--version! est présent dans les arguments du programme. Cette fonction aura le prototype suivant, et la fonction main est donnée

\ifprintanswers
\begin{solution}
\begin{code}[c]
void parse_arguments(int argc, char *argv[])
{
    for (int i = 0; i < argc; i++)
        if (strcmp(argv[i], "--version") == 0)
            printf("Version\n");
}

int main(int argc, char *argv[]) {
    parse_arguments(argc, argv);
}
\end{code}
\end{solution}
\else
\begin{code}[c]
void parse_arguments(int argc, char *argv[]);

int main(int argc, char *argv[]) {
    parse_arguments(argc, argv);
}
\end{code}
\fillwithdottedlines{8cm}
\fi

\question Écrire une fonction qui retourne la valeur minimale d'un tableau de réels. Le prototype est le suivant :

\ifprintanswers
\begin{solution}
\begin{code}[c]
double min(double array[], size_t size) {
    double min = array[0];
    for (int i = 1; i < size; i++)
        if (array[i] < min)
            min = array[i];
    return min;
}
\end{code}
\end{solution}
\else
\begin{code}[c]
double min(double array[], size_t size);
\end{code}
\fillwithdottedlines{8cm}
\fi

\end{questions}

\end{document}
