\documentclass[french,a4paper,addpoints,11pt]{exam}
\usepackage{hexercises}
\usepackage{mathtools, nccmath}
\title{Révision du cours Info1}
\seriesno{\texttt{0x20}}
\department{TIN}
\classroom{INFO2-TIN}

\setlength\answerlinelength{10 cm}
\setlength\answerskip{3ex}
\setlength\answerclearance{1.1ex}
\CorrectChoiceEmphasis{}
\renewcommand{\thepartno}{\alph{partno}}
\renewcommand{\partlabel}{\thepartno.}
\renewcommand{\arraystretch}{1.75} % expand the cells vertically

\begin{document}
\maketitle
\thispagestyle{headandfoot}

\begin{questions}

\question Considérez les déclarations suivantes :

\begin{lstlisting}
char c = 3;    short s = 7;
int i = 3;     long l = 4;
float f = 3.3; double d = 7.7;
\end{lstlisting}

Pour chacune des expressions ci-dessous, indiquez leur type et leur valeur.

\textbf{Exemple :} \CD{2 / 3 * c}, réponse : \CD{(int)0}.

\begin{parts}
    \part \CD{c / 2}
    \answerline[\CD{(int)1}]

    \part \CD{s + c / 10}
    \answerline[\CD{(int)7}]

    \part \CD{l + i / 2.0}
    \answerline[\CD{(double)5.5}]

    \part \CD{d + f}
    \answerline[\CD{(double)11}]

    \part \CD{(int)d + f}
    \answerline[\CD{(float)10.3}]

    \part \CD{(int)d + l}
    \answerline[\CD{(long)11}]

    \part \CD{c << 2}
    \answerline[\CD{(int)12}]

    \part \CD{s & 0xf0}
    \answerline[\CD{(int)0}]

    \part \CD{s && 0xf0}
    \answerline[\CD{(int)1}]

    \part \CD{d + f == s + l}
    \answerline[\CD{(int)0}]
\end{parts}

\question
Dans chacune des structures de contrôle ci-dessous, indiquer la nature de l'erreur.

\begin{parts}

\part ~\par

\begin{lstlisting}
double x = 100.0;
size_t i = 0;
do
    x = x / 2.0;
    i++;
while (x > 1.0);
\end{lstlisting}

\begin{solutionordottedlines}[2cm]
Il manque les accolades autour du bloc \CD{do..while}.
\end{solutionordottedlines}

\part ~\par
\begin{lstlisting}
long x = 100;
if (x = 0)
    printf("Erreur : la valeur 0 est interdite !\n");
\end{lstlisting}

\begin{solutionordottedlines}[2cm]
Le test d'égalité utilise l'opérateur \CD{==}. L'opérateur d'affectation \CD{=} n'est pas valable.
\end{solutionordottedlines}

\part ~\par
\begin{lstlisting}
double x = 100.0;
switch (x) {
    case 0:
        printf("x est nul.\n");
        break;
    default:
        print("OK.\n");
}
\end{lstlisting}

\begin{solutionordottedlines}[2cm]
L'instruction \CD{switch} n'est pas applicable à un type à virgule flottante.
\end{solutionordottedlines}

\part ~\par
\begin{lstlisting}
for (int i = 0; i < 10; i++);
{
    printf("%d\n", i);
}
\end{lstlisting}

\begin{solutionordottedlines}[2cm]
Le point virgule à la fin de l'instruction termine cette dernière. Le bloc formé des accolades n'appartient pas à la boucle.
\end{solutionordottedlines}

\part ~\par
\begin{lstlisting}
int i = 0;
while i < 100
{
    printf("%d\n", ++i);
}
\end{lstlisting}

\begin{solutionordottedlines}[2cm]
Il manque des parenthèses autour de la condition de l'instruction \CD{while}.
\end{solutionordottedlines}
\end{parts}

\newpage

\question
On s'intéresse ici au passage par adresse. Observez le programme suivant et indiquez ce que vous voyez sur la sortie standard.

\begin{lstlisting}
#include <stdio.h>
#include <stdlib.h>

int test(int a, int *b, int *c, int *d) {
    a = *b;
    *b = *b + 5;
    *c = a + 2;
    d = c;
    return *d;
}

int main() {
  int a = 0, b = 100, c = 200, d = 300, e = 400;
  e = test(a, &b, &c, &d);
  printf("%05d %d %d %d %d", a, b, c, d, e);
}
\end{lstlisting}

\begin{solutionordottedlines}[1cm]
\CD{00000 105 102 300 102}
\end{solutionordottedlines}

\question Algorithme sur les tableaux : écrire une fonction qui reçoit en paramètre un tableau d'entiers et qui retourne la position de la première occurrence d'une valeur dans ce tableau, ou \CD{-1} si la valeur n'est pas présente. Utiliser la syntaxe pointeur pour le paramètre tableau.

\ifprintanswers
\begin{solution}
\begin{lstlisting}
int index(int *array, size_t size, int value) {
  for (int i = 0; i < size; i++)
      if (array[i] == value)
          return i;
  return -1;
}
\end{lstlisting}
\end{solution}
\else
\fillwithdottedlines{5cm}
\fi

\question
Écrire une fonction qui calcul la longueur totale des segments de droite dont les points sont reçus en paramètre. Les données sont un tableau composé de N par 2. Les indices du sous tableau sont les coordonnées X et Y des points.

\ifprintanswers
\begin{solution}
\begin{lstlisting}
double distance(double x1, double y1, double x2, double y2) {
  return sqrt(pow(x2 - x1, 2) + pow(y2 - y1, 2));
}

double length(double array[][2], size_t size) {
  double length = 0;
  for (int i = 0; i < size - 1; i++)
      length += distance(
          array[i][0], array[i][1],
          array[i + 1][0], array[i + 1][1]
      );
  return length;
}
\end{lstlisting}
\end{solution}
\else
\fillwithdottedlines{7cm}
\fi

\end{questions}

\end{document}
